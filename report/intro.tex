%!TEX root = report.tex
A large number of systems change their macroscopic properties at thermal equilibria. For example magnetic atoms align them selves to form a magnetic material at low temperature or high pressure. When modeled mathematically, these phase transitions only occur in infinitely large systems \cite{kenzel1997physics}. This paper investigates a simulation of a finite system that models a system that exhibits macroscopic properties at its thermal equilibrium, the Ising ferromagnet to be exact.

\Cref{ss:intro:ising} introduces the Ising model of ferromagnetism, the next section discusses the Metropolis Monte Carlo method that is used to solve the Ising model numerically.

\subsection{Ising Model}
	\label{ss:intro:ising}
	%!TEX root = report.tex
% \todo[inline]{Ising Model in general}
A magnet can be modeled as a large collection of electronic spins. In the Ising model spins point either up or down, i.e. $S_i = \pm 1$ for $i = 1, \dotsc, N$, where $N$ represents the number of spins \cite{strogatz2014nonlinear}.

One important property of the modeled magnet is its magnetization $m$, which is defined as its average spin:
\begin{equation}
	m = \abs{\frac{1}{N} \sum_{i = 1}^{N} S_i},
\end{equation}
where $N$ is the number of spins. At high temperatures the spins point in random directions, consequently the magnetization is approximately zero. As the temperature decreases the magnetization remains near zero until a critical temperature is reached, at which the material magnetizes, i.e.\ $m \gg 0$. Due to the symmetry of the spin two ferromagnetic states are possible.

\Cref{sss:intro:ising:1D} and \ref{sss:intro:ising:2D} introduce the one- and two-dimensional Ising model, respectively. 

% The behaviour of an organization of spins is characterized by its average energy and specific heat. The energy of one arrangement of spins, \spinset is
% 	\begin{equation}
% 		E = - \sum_{n = 1}^{N - 1}S_nS_{n+1}.
% 	\end{equation}
% The average energy $U$ at temperature $T = \beta^{-1}$ is defined as
% 	\begin{equation}
% 		U = \frac{
% 			\displaystyle\sum_{\spinset} \exp\left[-\beta E\right]
% 		}{
% 			\displaystyle\sum_{\spinset} \exp\left[-\beta E\right]
% 		}.
% 	\end{equation}
% At temperature $T$ the specific heat, $C$, is defined as
% 	\begin{equation}
% 		C = \beta^2
% 		\left[ 
% 			\frac{
% 				\displaystyle\sum_{\spinset} E^2 \exp\left[-\beta E\right]
% 			}{
% 				\displaystyle\sum_{\spinset} \exp\left[-\beta E\right]
% 			}	
% 		\right].
% 	\end{equation}

\subsubsection{One-Dimensional Model}
	\label{sss:intro:ising:1D}
	\todo[inline]{Describe 1D Ising Model}
	\todo[inline]{Energy of a configuration}
	\todo[inline]{Average Energy}
	\todo[inline]{Specific heat per spin}
	\todo[inline]{Present an prove analytical solution}


\subsubsection{Two-Dimensional Model}
	\label{sss:intro:ising:2D}
	\todo[inline]{2D Ising Model}
	\todo[inline]{Energy of a configuration}
	\todo[inline]{Average energy}
	\todo[inline]{Average magnetization per spin}
	\todo[inline]{Specific heat}
	\todo[inline]{Present analytical solution}

\subsection{Metropolis Monte Carlo}
	\label{ss:intro:mmc}
	%!TEX root = report.tex
	\todo[inline]{Consider random states, won't work, since ....}
	\todo[inline]{Metropolis MC in general}

	\todo[inline]{Importance sampling}

	\todo[inline]{The Metropolis solution}

	\todo[inline]{Use the general average stuff to show the next two functions.}
	\todo[inline]{How to compute average energy in the simulation}
	\todo[inline]{How to compute average magnetization in the simulation}


\Cref{s:method} discusses how the Metropolis Monte Carlo method is used to solve the Ising model. In \cref{s:experiment} the run experiments are introduced, their results are presented in \cref{s:results}. \Cref{s:discussion} discusses the found results and \cref{s:conclusion} concludes this paper.