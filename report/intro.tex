%!TEX root = report.tex
A large number of systems change their macroscopic properties at thermal equilibria. For example magnetic atoms align them selves to form a magnetic material at low temperature or high pressure. When modeled mathematically, these phase transitions only occur in infinitely large systems \cite{kenzel1997physics}. This paper investigates a simulation of a finite system that models a system that exhibits macroscopic properties at its thermal equilibrium, the Ising ferromagnet to be exact.

\Cref{ss:intro:ising} introduces the Ising model of ferromagnetism, the next section discusses the Metropolis Monte Carlo method that is used to solve the Ising model numerically.

\subsection{Ising Model}
	\label{ss:intro:ising}
	%!TEX root = report.tex
% \todo[inline]{Ising Model in general}
A magnet can be modeled as a large collection of electronic spins. In the Ising model spins point either up, \mbox{\spin{} = +1}, or down, \mbox{\spin{} = -1} \cite{strogatz2014nonlinear}. The magnetization of a magnet is defined as its average spin:
\begin{equation*}
	\magnetization = \abs{\frac{1}{\numberOfSpins} \sum_{i = 1}^{\numberOfSpins} \spin{i}},
\end{equation*}
where \numberOfSpins is the number of spins. At high temperatures the spins point in random directions, consequently the magnetization is approximately zero. At a low enough temperature all spins in the two-dimensional model align themselves, this effect is called spontaneous magnetization. The temperature at which this phase transition occurs is called the critical temperature, \criticalTemperature \cite{cai20011Handout}

\Cref{sss:intro:ising:1D} and \ref{sss:intro:ising:2D} introduce the one- and two-dimensional Ising model, respectively. 

\subsubsection{One-Dimensional Model}
	\label{sss:intro:ising:1D}
	\textcite{ising1925beitrag} introduced a model consisting of a one-dimensional lattice op spin variables. Contrary to the two dimensional model this model does not exhibit state transitions. The Hamiltonian of the one dimensional Ising model with the set spins \mbox{\configuration{} = \spinset} is
	\begin{equation}\label{eq:intro:ising:1D:hamiltonian}
		\hamiltonian\left(\configuration{} \right) = - \interactionStrength \sum_{\pair{i}{j}}\spin{i}\spin{j} - \magneticMoment \sum_{i} \spin{i}.
	\end{equation}
	Where \pair{i}{j} is a nearest neighbour pair, the nearest neighbour of \spin{i} in the one dimensional model are \spin{i - 1} and \spin{i + 1}. \interactionStrength specifies the strength of the interactions between the particles. In a ferromagnetic model, \mbox{$\interactionStrength > 0$} neighboring spins prefer to be parallel. In the anti-ferromagnetic model, \mbox{$\interactionStrength < 0$} spins prefer a direction different to one of their neighbors. The constant \magneticMoment represents the external magnetic field, the spins want to align with the direction of $h$, \ie when \mbox{$h > 0$} spins prefer to be positive. 

	In the following the zero-field ferromagnetic model, \ie \mbox{$\interactionStrength = 1$} and \mbox{$\magneticMoment = 0$},  is considered. The energy \energy of a configuration of spins, \configuration{}, in this model is given by 
	\begin{equation*}
		\energy\left(\configuration{}\right) = \sum_{n = 1}^{\numberOfSpins - 1}\spin{n}\spin{n+1}.
	\end{equation*}
	The probability of a configuration of spins \configuration{} at temperature \temperature is given by
	\begin{equation}
		\label{eq:intro:1D:configurationProbability}
		P\left(\configuration{}\right) = 
		\frac{1}{\partitionfunction} \exp\left[-{E(\configuration{i})}\frac{1}{\temperature}\right],
	\end{equation}
	where $\temperature = \rfrac{1}{\beta}$ and \partitionfunction is the partition function:
	\begin{equation}
		\label{eq:intro:1D:partitionFunction}
		\partitionfunction = \sum_{\spinset} \exp\left[- \energy \beta \right].
	\end{equation}

	% START ANALYTICAL STUFF
	Both the one and two dimensional Ising model can be solved analytically. Under free end boundary conditions, \ie the boundary particles, \spin{1} and \spin{\numberOfSpins}, only observe one neighbor \cite{landau2014guide}, the analytical solution of \cref{eq:intro:1D:partitionFunction} is
	\begin{equation}
		\label{eq:introduced1D:partitionFunctionAnalyticalSolution}
		\partitionfunction = {(2 \cosh \beta)}^\numberOfSpins.
	\end{equation}
	\todo[inline]{Wat is $E_n$?}
	\todo[inline]{Waarover loopt die som?}
	The average energy in the system can be expressed as a function of \partitionfunction \cite{Murray20011Handout}
	\begin{equation*}
		% \label{eq:intro:1D:averageEnergyOriginal}
		\averageEnergy = \frac{1}{\partitionfunction} \cdot \sum_{n} \energy_n \cdot \exp\left[-\beta \energy_n \right].		
	\end{equation*} 
	Observing that
	\begin{equation*}
		\frac{\partial \partitionfunction}{\partial \beta} = \sum_{n} - \energy_n \exp\left[-\beta \energy_n \right],
	\end{equation*}
	and by following the steps presented in \cref{a:derivations:averageEnergy} it can be found that
	\begin{equation*}
		\averageEnergy = - \frac{\partial \ln \left[ \partitionfunction \right]}{\partial \beta} = - \numberOfSpins \cdot \tanh (\beta).
	\end{equation*}
	Consequently $\rfrac{\averageEnergy}{\numberOfSpins} = - \tanh (\beta)$.

	The specific heat describes how the average energy changes as a function of the temperature. Consequently
	\begin{equation*}
		\specificHeat = \frac{\partial \averageEnergy}{\partial \temperature} = \numberOfSpins {\left(\frac{\beta}{\cosh(\beta)}\right)}^{2}
	\end{equation*}
	as shown in \cref{a:derivations:specificHeat} \cite{warkHandout}, consequently
	\begin{equation*}
	 	\frac{\specificHeat}{\numberOfSpins} = {\left(\frac{\beta}{\cosh(\beta)}\right)}^2.
	 \end{equation*}
 
\subsubsection{Two-Dimensional Model}
	\label{sss:intro:ising:2D}
	The 2D Ising model is a square lattice whose lattice sites are occupied by spins. Each spin has either a positive or a negative spin \cite{kenzel1997physics}. The Hamiltonian of the 2D model is the same as the one of the one dimensional model given in \cref{eq:intro:ising:1D:hamiltonian}. The pairs of nearest neighbours are now found by looking at the four connected neighbours, \ie the nearest neighbours of \mbox{spin \spin{i,j}} are \spin{i - 1, j}, \spin{i + 1, j}, \spin{i, j - 1} and \spin{i, j + 1}. \todo{Wat als ie op de rand zit, lopen we rond, of zien we gewoon minder?}The energy of a configuration \configuration{n} that has $\dimensionality \times \dimensionality$ spins is computed as
	\begin{equation}
		\label{eq:intro:ising:2d:energy}
		\begin{split}
		\energy\left(\configuration{n} \right) 
			& = - \sum_{i = 1}^{\dimensionality - 1}\sum_{j = 1}^{\dimensionality} \spin{i,j}\spin{i+1,j}\\
			&\quad\quad - \sum_{i = 1}^{\dimensionality}\sum_{j = 1}^{\dimensionality - 1} \spin{i,j} \spin{i, j+1}.
		\end{split}
	\end{equation}

	% Analytical stuff
	The two-dimensional Ising model has been solved analytically by \textcite{onsager1944crystal}. He showed that the average magnetization per spin on a infinite 2D lattice, \ie $\numberOfSpins = \infty$, is
	\begin{equation}
		\label{eq:intro:ising:2D:magnetizationAnalytical}
		\frac{\magnetization}{\dimensionality^2} = \begin{cases}
			{(1 - {\sinh}^{-4}(2\beta))}^2 & \text{if } \temperature < \criticalTemperature\\
			0 								& \text{if } \temperature > \criticalTemperature
		\end{cases}
	\end{equation}
	where
	\begin{equation*}
		\criticalTemperature = \frac{2}{\ln \left(1 + \sqrt{2}\right)}.
	\end{equation*}

	Given \cref{eq:intro:1D:partitionFunction} solving the the Ising model is relatively simple. To find which configurations of spins result in an equilibirium one only needs to try them all. Unfortunately the computational complexity of this operation is exponential in $\numberOfSpins$, the number of spins. To be exact, a lattice with \numberOfSpins spins has $2^\numberOfSpins$ possible configurations, computing \energy according to \cref{eq:intro:ising:2d:energy} for one configuration takes $2\numberOfSpins$ steps. This leads to $2\numberOfSpins2^\numberOfSpins$ computation steps \cite{kenzel1997physics}. Solving the problem with the Metropolis Monte Carlo method circumvents this complexity problem. 

\subsection{Metropolis Monte Carlo}
	\label{ss:intro:mmc}
	%!TEX root = report.tex

Monte Carlo methods rely on random sampling to obtain numerical results. They are often used to solve problems that might be deterministic in principle but are difficult to solve with other approaches. One of the applications of Monte Carlo experiments is sampling, \ie generating draws from some probability distribution \cite{kroese2014monte}. 

In the context of the Ising model one could naively consider using a few randomly generated states to compute the partition function. However the central limit theorem tells us that these states have an energy that is approximately $\bigOh{\sqrt{\numberOfSpins}}$ for sufficiently large $N$. However the states that we are interested in have an energy of the order \bigOh{\numberOfSpins}, which means that they are not generated at all by the naive method. 

Consequently we need some way to generate the physically relevant states. This can be done by relaxing some configuration into a thermal equilibrium by generating from it a new sequence of states. This requires a transition probability $\transitionProbability\left(\configuration{i} \to \configuration{j} \right)$ from configuration \configuration{i} to configuration \configuration{j}. In thermal equilibrium the probability of finding a given configuration is presented in \cref{eq:intro:1D:configurationProbability}. As we require $P\left(\configuration{i}\right)$ to be stationary in thermal equilibrium we get the detailed balance:
\begin{multline}\label{eq:intro:mcm:detailedbalance}
	\transitionProbability\left(\configuration{i} \to \configuration{j}\right)\exp\left[-\energy(\configuration{i})\beta \right] = \\
	\transitionProbability\left(\configuration{j} \to \configuration{i}\right)\exp\left[-\energy(\configuration{j})\beta \right].
\end{multline}
The function $\transitionProbability(\cdot)$ needs to cover the entire configuration space. The Metropolis algorithm is one of the algorithms that ensures this \cite{kenzel1997physics}. 

This algorithm starts in some initial configuration, it then moves to subsequent configurations by flipping one randomly selected spin with a probability defined by \mbox{$\transitionProbability(\cdot)$}. This is repeated for a given number of steps. Generally one should give the system as number of steps to relax into an interesting state before actually using the generated states.

\Cref{s:method} discusses how the Metropolis Monte Carlo method is used to solve the Ising model. In \cref{s:experiment} the run experiments are introduced, their results are presented in \cref{s:results}. \Cref{s:discussion} discusses the found results and \cref{s:conclusion} concludes this paper.