%!TEX root = report.tex
This section introduces the experiments we ran with the model introduced in the previous sections. In the experiments below the number of iterations, \numberOfIterations, is not necessarily equal to \numberOfSpins, the number of spins, but set independently. To give the system time to relax into the interesting states we perform \mbox{$\rfrac{1}{10} \cdot \numberOfIterations$} Monte Carlo steps, before actually taking the samples used to compute the results. 

 The experiments we ran with the one and two dimensional model are discussed in \cref{ss:experiment:1D,ss:experiment:2D}, respectively.

\subsection{One-Dimensional Model}
\label{ss:experiment:1D}
	In the 1D model we are interested in both the average energy and the specific heat per spin in the following parameter space \mbox{$\temperature = 0.2, 0.4, \dotsc, 4$}, \mbox{$\numberOfSpins=10,100,1000$} and \mbox{$\numberOfIterations=1000,10000$}. 

	\averageEnergy, the average energy is given by
	\begin{equation}\label{eq:experiment:1D:averageEnergy}
		\averageEnergy = \frac{1}{\#\sampleSet} \sum_{\configuration{i} \in \sampleSet} E(\configuration{i}),
	\end{equation}
	where $\sampleSet = \left\{ \configuration{1}, \dotsc, \configuration{\numberOfSamples} \right\}$ is the set of configurations generated during the Monte Carlo steps. \specificHeat, the specific heat is defined as
	\begin{equation}\label{eq:experiment:1D:specificHeat}
		\specificHeat = \beta^2 \left( 
			\frac{1}{\#\sampleSet} \left(  \sum_{\configuration{i} \in \sampleSet} E^2(\configuration{i})\right) - \averageEnergy^2 
		\right).
	\end{equation}
	Furthermore we will compare the results of the simulation with the analytical solution presented earlier. To compare the numerical and analytical results the mean accuracy of the specific heat and the average energy per spin are computed. The accuracy of a variable where $\nu$ and $\alpha$ represent the numerically and analytically found values, respectively is
	\begin{equation}\label{eq:experiment:1D:accuracy}
		\accuracy = 1 - \abs{ \frac{\abs{\nu - \alpha}}{\alpha}}.
	\end{equation}
	The dicsussed experiment is implemented in \cref{lst:experiment:1d}. 
	The implementation of \cref{eq:experiment:1D:averageEnergy,eq:experiment:1D:specificHeat,eq:experiment:1D:accuracy}
	are presented in \cref{lst:statistics:computeAverageEnergy,lst:statistics:computeSpecificHeat,lst:statistics:computeAccuracy}, respectively. 
	All mentioned listings can be found in \cref{a:implementation}.	
	

\subsection{Two-Dimensional Model}
\label{ss:experiment:2D}
	In the 2D model we are not only interested in the average energy and specific heat per spin but also the average magnetization per spin. \todo{Formule toevoegen?} The following parameter space is used: \mbox{$\temperature = 0.2, 0.4, \dotsc, 4$}, \mbox{\numberOfSpins = 10, 50, 100} and $\numberOfIterations = 1000, 10000$. The found average average magnetization per spin is compared with the analytical solution presented in \cref{eq:intro:ising:2D:magnetizationAnalytical}.

	\todo[inline]{Refer to listings/section in appendix with the code for this experiment.}

