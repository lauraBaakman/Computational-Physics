%!TEX root = report.tex
This section introduces the experiments we ran with the model introduced in the previous sections. The required random numbers were generated with a Mersenne Twister with seed 0. To give the system time to relax into the interesting states we perform \mbox{$\rfrac{1}{10} \cdot \numberOfIterations$} Monte Carlo steps, before actually taking the samples used to compute the results. 

 The experiments we ran with the one and two dimensional model are discussed in \cref{ss:experiment:1D,ss:experiment:2D}, respectively.

\subsection{One-Dimensional Model}
\label{ss:experiment:1D}
	In the one-dimensional model we are interested in both the average energy and the specific heat per spin in the following parameter space \mbox{$\temperature = 0.2, 0.4, \dotsc, 4$}, \mbox{$\numberOfSpins=10,100,1000$} and \mbox{$\numberOfIterations=1000,10000$}. 

	\averageEnergy, the average energy is given by
	\begin{equation}\label{eq:experiment:1D:averageEnergy}
		\averageEnergy = \frac{1}{\#\sampleSet} \sum_{\configuration{i} \in \sampleSet} E(\configuration{i}),
	\end{equation}
	where $\sampleSet = \left\{ \configuration{1}, \dotsc, \configuration{\numberOfSamples} \right\}$ is the set of configurations generated during the Monte Carlo steps. \specificHeat, the specific heat is defined as
	\begin{equation}\label{eq:experiment:1D:specificHeat}
		\specificHeat = \beta^2 \left( 
			\frac{1}{\#\sampleSet} \left(  \sum_{\configuration{i} \in \sampleSet} E^2(\configuration{i})\right) - \averageEnergy^2 
		\right).
	\end{equation}
	Furthermore we will compare the results of the simulation with the analytical solution presented in \cref{ss:intro:ising}. To compare the numerical and analytical results the mean accuracy of the specific heat and the average energy per spin are computed. The accuracy of a variable where $\nu$ and $\alpha$ represent the numerically and analytically found values, respectively, is
	\begin{equation}\label{eq:experiment:1D:accuracy}
		\accuracy = 1 - \abs{ \frac{\abs{\nu - \alpha}}{\alpha}}.
	\end{equation}

	The discussed experiment is implemented in \cref{lst:experiment:1d}. 
	The implementation of \cref{eq:experiment:1D:averageEnergy,eq:experiment:1D:specificHeat,eq:experiment:1D:accuracy}
	are presented in \cref{lst:statistics:computeAverageEnergy,lst:statistics:computeSpecificHeat,lst:statistics:computeAccuracy}, respectively. 
	All mentioned listings can be found in \cref{a:implementation}.	
	

\subsection{Two-Dimensional Model}
\label{ss:experiment:2D}
	The two-dimensional model should exhibit a phase transition. Consequently we are not only interested in the average energy and specific heat per spin but also the average magnetization per spin. The magnetization of the Ising model can be computed as:
	\begin{equation}\label{eq:experiment:2D:magnetization}
		\magnetization = \frac{1}{\#\sampleSet} \sum_{\configuration{i} \in \sampleSet} \sum_{\spin{j} \in \configuration{i}} \spin{j}.
	\end{equation}
	The following parameter space is used: $\mbox{\temperature = 0.2,} 0.4, \dotsc, 4$, $\mbox{\dimensionality = 10,} 50, 100$ and $\mbox{\numberOfIterations = 1000,} 10000$. The found average average magnetization per spin is compared with the analytical solution presented in \cref{eq:intro:ising:2D:magnetizationAnalytical}.

	The discussed experiment is presented in \cref{lst:experiment:2d} in \cref{a:implementation}.