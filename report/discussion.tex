%!TEX root = report.tex
In this section the results of the tow experiments, presented in \cref{s:results}, are discussed.

\subsection{One-Dimensional Model}
	
	We have observed that the accuracy increases as the temperature increases and that at very low temperatures the accuracy is very low. This is caused by the Boltzman factor, which is low when the \temperature is low. Thus very few potential states are accepted when the temperature is low. Consequently the system `moves slower' and takes longer to end up in the interesting states.
	
	The results in \cref{a:results1D} showed that the accuracy increases as the number of samples increases. The cause for this is that the analytical solutions are based on systems with an infinite number of spins. Consequently system with more spins are closer to systems with an infinite number of spins, and their results are thus closer to those of the infinite system. 

	Another factor that increases the accuracy of the system is the number of samples. Firstly more samples means that the system gets more time to relax. Secondly a higher number of samples makes decreases the influence of outlier states. Lastly the probability of ending up in a interesting configuration increases as the number of spins flipped increases.

	As expected we do not observe a phase transition in the one-dimensional model. 

\subsection{Two-Dimensional Model}
	\todo[inline]{Interpret results in terms of a phase transition from a state with magnetization zero to a state with definite magnetization (slide 31)}
	\todo[inline]{Blaat uit p 192 physics by computer.}


