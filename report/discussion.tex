%!TEX root = report.tex
In this section the results of the tow experiments, presented in \cref{s:results}, are discussed.

\subsection{One-Dimensional Model}
	
	We have observed that the accuracy increases as the temperature increases and that at very low temperatures the accuracy is very low. This is caused by the Boltzman factor, which is low when the \temperature is low. Thus very few potential states are accepted when the temperature is low. Consequently the system `moves slower' and takes longer to end up in the interesting states.
	
	The results in \cref{a:results1D} showed that the accuracy increases as the number of samples increases. The cause for this is that the analytical solutions are based on systems with an infinite number of spins. Consequently system with more spins are closer to systems with an infinite number of spins, and their results are thus closer to those of the infinite system. 

	Another factor that increased the accuracy of the system is the 

	\todo[inline]{Waarom wordt de accuracy beter als het aantal samples hoger wrodt}
	% When the number of samples increase, the accuracy of the MMC also increases. This is to be expected as the it has a higher probability of ending up in a better configuration as more trials are performed.	

	
	

	\todo[inline]{Er is geen phase transition, }
	% For sufficiently low temperatures, the system exhibits magnetic properties even in absence of an external magnetic field. That is, neighboring spins interact on each other in such a way that they produce a magnetic moment. This is true for the one-dimensional Ising model. It is then interesting to see for which temperature the system undergoes a phase transition from an ordered, spontaneously magnetized system to a disordered system that has no magnetic properties – unless activated by an external magnetic field. However, Ising has proven that the one-dimensional system does not undergo a phase transition[2] and we see no such transition in our results.	


\subsection{Two-Dimensional Model}
	\todo[inline]{Interpret results in terms of a phase transition from a state with magnetization zero to a state with definite magnetization (slide 31)}
	\todo[inline]{Blaat uit p 192 physics by computer.}


