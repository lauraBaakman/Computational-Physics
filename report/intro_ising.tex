%!TEX root = report.tex
% \todo[inline]{Ising Model in general}
A magnet can be modeled as a large collection of electronic spins. In the Ising model spins point either up, \mbox{\spin{n} = +1}, or down, \mbox{\spin{n} = -1} \cite{strogatz2014nonlinear}. The magnetization of a magnet is defined as its average spin:
\begin{equation}
	m = \abs{\frac{1}{N} \sum_{i = 1}^{N} \spin{i}},
\end{equation}
where $N$ is the number of spins. At high temperatures the spins point in random directions, consequently the magnetization is approximately zero. At a low enough temperature all spins in the two-dimensional model align themselves, this effect is called spontaneous magnetization. The temperature at which this phase transition occurs is called the critical temperature, \criticalTemperature \cite{cai20011Handout}

\Cref{sss:intro:ising:1D} and \ref{sss:intro:ising:2D} introduce the one- and two-dimensional Ising model, respectively. 

\subsubsection{One-Dimensional Model}
	\label{sss:intro:ising:1D}
	\textcite{ising1925beitrag} introduced a model consisting of a one-dimensional lattice op spin variables. Contrary to the two dimensional model this model does not exhibit state transitions. In the one-dimensional ferromagnetic zero field model, the model that is considered here, each spin interacts only with its nearest neighbors, i.e. \spin{n} interacts only with \spin{n-1} and \spin{n+1}.

	The energy $E$ of a particular arrangement of spins, \spinset, under free end boundary conditions is
	\begin{equation}
		\label{eq:intro:1D:energy}
		E(\spinset) = - \sum_{n = 1}^{N - 1}\spin{n}\spin{n+1}.
	\end{equation}
	Due to the use of the free end boundary condition particles the first and the last spin of the lattice, i.e. \spin{1} and \spin{N} see no neighbor on one side \cite{landau2014guide}.
	The probability of a configuration of spins \configuration{i} at temperature $T$ is given by
	\begin{equation}
		\label{eq:intro:1D:configurationProbability}
		P\left(\configuration{i}\right) = 
		\frac{1}{\partitionfunction} \exp\left[{E(\configuration{i})}\frac{1}{T}\right],
	\end{equation}
	where $T = \rfrac{1}{\beta}$ and $Z$ is the partition function:
	\begin{equation}
		\label{eq:intro:1D:partitionFunction}
		\partitionfunction = \sum_{\spinset} \exp\left[- E \beta \right],
	\end{equation}
	where $\beta = \rfrac{1}{T}$.
	The partition function sums over all possible configurations: $Z = \exp\left[-\beta E_1\right] + \exp\left[-\beta E_2\right] + \dotsb + \exp\left[-\beta E_N\right]$. 

	\todo[inline]{Analytical Solution $Z$}
	\todo[inline]{Analytical Solution $U/N$}
	\todo[inline]{Analytical Solution $C/N$}


	\todo[inline]{Formules die ergens genoemd moeten worden.}
	The free energy $U$ is given by
	\begin{equation}
		U = \frac{1}{\partitionfunction} \cdot \sum_{\spinset} E \cdot \exp\left[-\beta E\right].		
	\end{equation} It can be shown that this is equivalent to \todo{referentie}
	\begin{equation}
		U = - \tanh \left( \beta \right) \cdot N.
	\end{equation}
	% 
	The specific heat $C$ is defined as
	\begin{equation}
		C = \frac{\beta^2}{Z} \cdot \left[ \sum_{\spinset} E^2 \cdot \exp\left[-\beta E\right]\right] - U^2,
	\end{equation}
	which can also be computed as: \todo{referentie}
	\begin{equation}
		C = \frac{1}{N} \left( \frac{\beta}{\cosh \beta} \right)^2.
	\end{equation}
	\todo[inline]{Present an prove analytical solution}
 
\subsubsection{Two-Dimensional Model}
	\label{sss:intro:ising:2D}
	\todo[inline]{2D Ising Model}
	\todo[inline]{Energy of a configuration}
	\todo[inline]{Average energy}
	\todo[inline]{Average magnetization per spin}
	\todo[inline]{Specific heat}
	\todo[inline]{Present analytical solution}