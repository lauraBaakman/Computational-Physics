%!TEX root = report.tex
% \todo[inline]{Ising Model in general}
A magnet can be modeled as a large collection of electronic spins. In the Ising model spins point either up, \mbox{\spin{n} = +1}, or down, \mbox{\spin{n} = -1} \cite{strogatz2014nonlinear}. The magnetization of a magnet is defined as its average spin:
\begin{equation}
	m = \abs{\frac{1}{N} \sum_{i = 1}^{N} \spin{i}},
\end{equation}
where $N$ is the number of spins. At high temperatures the spins point in random directions, consequently the magnetization is approximately zero. At a low enough temperature all spins in the two-dimensional model align themselves, this effect is called spontaneous magnetization. The temperature at which this phase transition occurs is called the critical temperature, \criticalTemperature \cite{cai20011Handout}

\Cref{sss:intro:ising:1D} and \ref{sss:intro:ising:2D} introduce the one- and two-dimensional Ising model, respectively. 

\subsubsection{One-Dimensional Model}
	\label{sss:intro:ising:1D}
	\textcite{ising1925beitrag} introduced a model consisting of a one-dimensional lattice op spin variables. Contrary to the two dimensional model this model does not exhibit state transitions. The Hamiltonian of the one dimensional Ising model with the set spins \mbox{\configuration{i} = \spinset} is
	\begin{equation}
		\hamiltonian = - \interactionStrength \sum_{n = 1}^{N - 1}\spin{n}\spin{n+1} - \magneticMoment \sum_{n = 1}^{N} \spin{n}.
	\end{equation}
	\interactionStrength specifies the strength of the interactions between the particles. In a ferromagnetic model, \mbox{$\interactionStrength > 0$} neighboring spins prefer to be parallel. In the anti-ferromagnetic model, \mbox{$\interactionStrength < 0$} spins prefer a direction different to one of their neighbors. The constant \magneticMoment represents the external magnetic field, the spins want to align with the direction of $h$, \ie when \mbox{$h > 0$} spins prefer to be positive. 

	In the following the zero-field ferromagnetic model, \ie \mbox{$\interactionStrength = 1$} and \mbox{$\magneticMoment = 0$},  is considered. The energy \energy of a configuration of spins, \mbox{$\configuration{i} = \spinset$}, in this model is given by 
	\begin{equation}
		\energy\left(\configuration{i}\right) = \sum_{n = 1}^{N - 1}\spin{n}\spin{n+1}.
	\end{equation}
	The probability of a configuration of spins \configuration{i} at temperature \temperature is given by
	\begin{equation}
		\label{eq:intro:1D:configurationProbability}
		P\left(\configuration{i}\right) = 
		\frac{1}{\partitionfunction} \exp\left[{E(\configuration{i})}\frac{1}{\temperature}\right],
	\end{equation}
	where $\temperature = \rfrac{1}{\beta}$ and \partitionfunction is the partition function:
	\begin{equation}
		\label{eq:intro:1D:partitionFunction}
		\partitionfunction = \sum_{\spinset} \exp\left[- \energy \beta \right].
	\end{equation}

	Both the one and two dimensional Ising model can be solved analytically. Under free end boundary conditions, \ie the boundary particles, \spin{1} and \spin{N}, only observe one neighbor \cite{landau2014guide}, the analytical solution of \cref{eq:intro:1D:partitionFunction} is
	\begin{equation}
		\label{eq:introduced1D:partitionFunctionAnalyticalSolution}
		\partitionfunction = {(2 \cosh \beta)}^N.
	\end{equation}
	The average energy in the system can be expressed as a function of \partitionfunction \cite{Murray20011Handout}
	\begin{equation}
		\label{eq:intro:1D:averageEnergyOriginal}
		\averageEnergy = \frac{1}{\partitionfunction} \cdot \sum_{n} \energy_n \cdot \exp\left[-\beta \energy_n \right].		
	\end{equation} 
	Observing that
	\begin{equation}
		\frac{\partial \partitionfunction}{\partial \beta} = \sum_{n} - \energy_n \exp\left[-\beta \energy_n \right],
	\end{equation}
	following the steps presented in \cref{a:derivations:averageEnergy} we find that
	\begin{equation}
		\averageEnergy = - \frac{\partial \ln \left[ \partitionfunction \right]}{\partial \beta} = - N \cdot \tanh (\beta).
	\end{equation}
	Consequently $\rfrac{U}{N} = - \tanh (\beta)$.

	The specific heat, \specificHeat, can be computed similarly to the average energy. This variable describes how the energy changes as a function of the temperature. Consequently it can be found by differentiating the internal energy with respect to the temperature \cite{warkHandout}:
	\todo[inline]{Add derivation to the appendix.}
	\begin{equation}
		\specificHeat 
		= \frac{\partial \averageEnergy}{\partial \temperature} 
		= - \frac{\beta}{\temperature} \frac{\partial \averageEnergy}{\partial \beta} 
		= - \beta^2 \frac{\partial \averageEnergy}{\partial \beta}.
	\end{equation}
	Therefore
	\begin{equation}
		\frac{C}{N} = - \beta^2 \frac{\partial \averageEnergy}{\partial \beta}  = {\left( \frac{\beta}{\cosh \beta} \right)}^2
	\end{equation}
 
\subsubsection{Two-Dimensional Model}
	\label{sss:intro:ising:2D}
	\todo[inline]{2D Ising Model}
	\todo[inline]{Energy of a configuration}
	\todo[inline]{Average energy}
	\todo[inline]{Average magnetization per spin}
	\todo[inline]{Specific heat}
	\todo[inline]{Present analytical solution}