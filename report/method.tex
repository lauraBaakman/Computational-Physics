%!TEX root = report.tex
\todo[inline]{What are we going to discuss in this section?}

\todo[inline]{Pseudocode of MMC}
%!TEX root = report.tex		
\begin{algorithm}
	\setstretch{1.2}
	\SetAlgoShortEnd
	\DontPrintSemicolon
	\SetKwInOut{Input}{input}\SetKwInOut{Output}{output}
	\Input{
		$\configuration{\mathit{init}}$ the initial configuration\newline
		$\numberOfIterations$ number of iterations
	}
	\BlankLine
	$\configuration{\mathit{cur}}$ := $\configuration{\mathit{init}}$\;
	\For{$i = 0$ \KwTo \numberOfIterations}{
		\spin{} := \FuncSty{selectRandomSpin(\configuration{\mathit{cur}})}\;
		$\configuration{\mathit{pot}}$ := \FuncSty{flipSpin(\spin{}, $\configuration{\mathit{cur}}$)}\;
		$\configuration{\mathit{cur}}$ := \FuncSty{select(
			$\configuration{\mathit{cur}}$,
			$\configuration{\mathit{pot}}$
		)}
	}
	\caption{$\text{MMC}(\configuration{\mathit{init}}, \numberOfIterations)$\label{alg:method:mmc}}
\end{algorithm}

\todo[inline]{Discuss selectRandomSpinFunction}

\todo[inline]{Discuss flipSpinConfigurationFunction}

\todo[inline]{Present selectNextConfigurationFunction pseudo code}
%!TEX root = report.tex		
\begin{algorithm}
	\setstretch{1.2}
	\SetAlgoShortEnd
	\DontPrintSemicolon
	\SetKwInOut{Input}{input}
	\SetKwInOut{Output}{output}
	\Input{
			$\configuration{\mathit{cur}}$ the current configuration \newline
			$\configuration{\mathit{pot}}$ the potential configuration
	}
	\Output{
		$\configuration{\mathit{new}}$ the selected configuration
	}

	\BlankLine
	$\Delta \energy$ := \FuncSty{computeDeltaE(
		$\configuration{\mathit{cur}}$, 
		$\configuration{\mathit{pot}}$
	)}\;
	$\xi$ := $\exp\left[- \beta\Delta\energy \right]$\;
	$\theta$ := \FuncSty{randomNumber($0$, $1$)}\;
	\leIf{$\xi > \theta$}{
		$\configuration{\mathit{new}}$ := $\configuration{\mathit{pot}}$\newline
	}{
		$\configuration{\mathit{new}}$ := $\configuration{\mathit{cur}}$
	}
	\caption{$\text{select}(\configuration{\mathit{cur}}, \configuration{\mathit{pot}})$\label{alg:method:selectConfig}}
\end{algorithm}

\todo[inline]{Discuss computeEnergyDifference function}
\todo[inline]{Determine W: p194 physics by computer}
\todo[inline]{Determine Delta E: p194 physics by computer}
% Bepaal w p194 physics by computer
% We only allow transitions at which at most one spin is changed. Furthermore if the change in spin results in an energetically better state, the transition to this state always happens, \ie $\transitionProbability\left(\configuration{i} \to \configuration{j}\right) = 1$. Using \cref{eq:intro:mcm:detailedbalance,eq:intro:1D:configurationProbability} gives the transition function for all other cases:
% \todo{Boltzman constant k_b = 1 eruit werken}
% 	\begin{equation*}
% 		\transitionProbability\left(\configuration{i} \to \configuration{j}\right) = \exp \left[\frac{\hamiltonian\left(\configuration{i}\right) - \hamiltonian\left(\configuration{j}\right)}{k_b \temperature}\right].
% 	\end{equation*}
% Writing \cref{eq:intro:ising:1D:hamiltonian} as
% 	\begin{equation*}
% 			\hamiltonian\left(\configuration{}\right) = - \spin{i} \sum_{j \in \neighbourhood{\spin{i}}} \spin{j} + \mathit{remainder}
% 	\end{equation*}
% where the sum is restricted to the nearest neighbors of \spin{i} and \textit{remainder} contains the part of \hamiltonian that does not depend on \spin{i} we can write 
% 	\begin{equation}
% 	\Delta\energy = 	
% 	\end{equation}
%

\todo[inline]{Discuss the if else}

The implementation of the presented algorithm can be found in \cref{a:implementation}. \todo[inline]{Refer to exact listings.}